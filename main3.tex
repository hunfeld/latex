\newcommand{\Klasse}{8b}
\newcommand{\Fach}{Mathematik}
\newcommand{\Nr}{2}
\newcommand{\Datum}{14.01.2022}
\newcommand{\mitlsg}{nein}
\newcommand{\schreibschrift}{nein}

%\input{praeambel}


\documentclass[addpoints,a4paper,ngerman,12pt]{exam}
\usepackage[ngerman]{babel}
\usepackage[a4paper,top=2cm,bottom=2cm,left=1cm,right=1cm]{geometry}
\usepackage[utf8]{inputenc}
\usepackage[T1]{fontenc}
\usepackage{booktabs}
\usepackage{graphicx}
\usepackage{csquotes}
\usepackage{paralist}
\usepackage{amsfonts, amssymb, amsmath}
\usepackage{tikz}
\usetikzlibrary{through,calc}


%ExamÜbersetzung
\pointpoints{Punkt}{Punkte}
\bonuspointpoints{Bonuspunkt}{Bonuspunkte}
\renewcommand{\solutiontitle}{\noindent\textbf{Lösung:}\enspace}
\chqword{Aufgabe}   
\chpgword{Seite} 
\chpword{Punkte}   
\chbpword{Bonus Punkte} 
\chsword{Erreicht}   
\chtword{Gesamt}
\checkboxchar{\Square}
\checkedchar{\CheckedBox}

\ifthenelse{\equal{\schreibschrift}{ja}}{\newcommand{\schrift}{\usefont{T1}{wela}{m}{n}}}{\newcommand{\schrift}{}}

%ExamHeader
\ifthenelse{\equal{\mitlsg}{ja}}{\printanswers}{}
%\printanswers
\ifthenelse{\equal{\mitlsg}{ja}}{\pagestyle{empty}}{ \pagestyle{headandfoot}} %headandfoot oder empty
\runningheadrule
\firstpageheader{\schrift
Gymnasium Dörpen\\ \Klasse \  - \Fach}{\schrift \hspace*{-4cm} Klassenarbeit Nr. \Nr \\ \hspace*{-4cm} am \Datum }{\schrift Name: \hspace*{70mm}}
\runningheader{\schrift Seite \thepage \  von \pageref{lastpage}}{\schrift \Fach \ Klassenarbeit Nr. \Nr }{\schrift Initialen:\hspace*{15mm}}
\firstpagefooter{}{}{\iflastpage{\schrift Viel Erfolg!}{\schrift Bitte umblättern!}}%{\thepage\,/\,\numpages}
\runningfooter{}{}{\iflastpage{\schrift Viel Erfolg!}{\schrift Bitte umblättern!}}%{\thepage\,/\,\numpages}}

%SchulzDef
\def\wu#1{\sqrt{{#1}\!\;\,}^{\!\;\!\rule[-.1ex]{.04em}{.5ex}}\;\!}
%  widetilde : \schl
\def\schl#1{\ifthenelse{\equal{#1}{i}}{\widetilde{\imath}}{
		\ifthenelse{\equal{#1}{j}}{\widetilde{\jmath}}{\widetilde{#1}} } } 

%  equations
\def\be#1{\begin{equation} \label{#1}}  \def\ee{\end{equation}}
\def\bea#1{\begin{eqnarray} \label{#1}}  \def\eea{\end{eqnarray}} 

%  greek
\let\a=\alpha   \let\b=\beta   \let\d=\delta  \let\g=\gamma
\let\l=\lambda  \let\o=\omega  \let\s=\sigma  \let\z=\zeta
\let\e=\varepsilon  \let\ph=\varphi   \let\ta=\vartheta
\let\vrho=\varrho   \let\D=\Delta     \let\G=\Gamma  
\let\L=\Lambda      \let\O=\Omega     \let\Y=\Upsilon


%  dotted vectors : \pvc , \ppvc
\def\pvc#1{{\buildrel {_{\hbox{\,\bf .}}} \over {{\vc #1}}}}
\def\ppvc#1{{\buildrel {_{\hbox{\,\bf ..}}} \over {{\vc #1}}}}

%  vectors : \vc
\def\pueb#1#2{\hspace*{-#1mm}{\buildrel{\hspace*{#1mm}
			\hspace*{#1mm} _\rightharpoonup}\over{#2}}\hspace*{-#1mm}
	\hspace*{-#1mm}\hspace*{-.2mm}}
\def\gpueb#1#2{\hspace*{-#1mm}\hspace*{.4mm}{\buildrel{
			\hspace*{#1mm}\hspace*{#1mm}{\displaystyle _\rightharpoonup}}
		\over{#2}}\hspace*{-#1mm}\hspace*{-#1mm}\hspace*{.4mm}}
\def\vc#1{
	\ifthenelse{\equal{#1}{f} \or \equal{#1}{d}}{\pueb{.4}{#1}}{
		\ifthenelse{\equal{#1}{e} \or \equal{#1}{k} \or \equal{#1}{l} 
			\or \equal{#1}{r} \or \equal{#1}{s} \or \equal{#1}{t} 
			\or \equal{#1}{\b} \or \equal{#1}{\beta} \or \equal{#1}{\ell} 
			\or \equal{#1}{\phi} \or \equal{#1}{\vrho} 
			\or \equal{#1}{\varrho}}{\pueb{.2}{#1}}{   
			\ifthenelse{\equal{#1}{j}}{\pueb{.4}{\jmath}}{
				\ifthenelse{\equal{#1}{i}}{\pueb{.4}{\imath}}{
					\ifthenelse{\equal{#1}{m} \or \equal{#1}{w} \or \equal{#1}{\o}
						\or \equal{#1}{\omega}}{\gpueb{0}{#1}}{
						\ifthenelse{\equal{#1}{A} \or \equal{#1}{R}}{\gpueb{.2}{#1}}{
							\ifthenelse{\equal{#1}{B} \or \equal{#1}{C} \or \equal{#1}{D} 
								\or \equal{#1}{E} \or \equal{#1}{F} \or \equal{#1}{G} \or \equal{#1}{H} 
								\or \equal{#1}{I} \or \equal{#1}{J} \or \equal{#1}{K} \or \equal{#1}{L} 
								\or \equal{#1}{M} \or \equal{#1}{N} \or \equal{#1}{O} \or \equal{#1}{P} 
								\or \equal{#1}{Q} \or \equal{#1}{S} \or \equal{#1}{T} \or \equal{#1}{U} 
								\or \equal{#1}{V} \or \equal{#1}{W} \or \equal{#1}{X} \or \equal{#1}{Y} 
								\or \equal{#1}{Z}}{\gpueb{.4}{#1}}{\pueb{0}{#1}}} }}} }} }
\newcommand{\lsgend}{\end{solutionorgrid}}
\newcommand{\lsg}[1][1]{\begin{solutionorgrid}[\stretch{#1}]}

%\ifthenelse{\equal{#1}{0}}{\begin{solutionorgrid}[\stretch{3}]}{
\newcommand{\inputt}[2][1]{\renewcommand{\lsg}[#1][1]{\begin{solutionorgrid}[\stretch{#1}]} \input{#2}}
	%\ifthenelse{\equal{#1}{0}}{\begin{solutionorgrid}[\stretch{3}]}{
	\newcommand{\seitenwechsel}{\ifthenelse{\equal{\mitlsg}{ja}}{}{\newpage}}
	%Längen Hf
	\newcommand{\cm}{ ~\text{cm}}
	\newcommand{\m}{ ~\text{m}}
	\newcommand{\km}{ ~\text{km}}
\begin{document}
	\schrift
	\parindent0cm
%	\ifprintanswers 	\else 	\makebox[\linewidth]{Name:\enspace \hrulefill\\}	\fi
	\ifprintanswers 	\else 	\vspace*{-15mm}\hrulefill	\fi

\qformat{ \textbf{Aufgabe \thequestion\ } (\thepoints)\hfill } %Fragennummerierung und Punktangabe	

	\begin{questions}
\question[8] Löse die folgenden Gleichungen durch Äquivalenzumformungen. Gib jeweils die Lösungsmenge an.\\ Mache \textbf{nur für b und c} die Probe.
\begin{parts}
	\part $7x-2=12$
	\lsg[1]
\begin{align*}
7x-2&=12&&|+2\\
7x&=14&&|:7\\
x&=2&& \\
\mathbb{L}&=\{2\}&&
\end{align*}
	\lsgend
	
	\part $-p-20=-6p-30$ %-28
	\lsg[1]
\begin{align*}
-p-20&=-6p-30&&|+20\\
-p&=-6p-10&&|+6p\\
5p&=-10&&|:5\\
p&=-2&& \\
\mathbb{L}&=\{-2\}&&
\end{align*}
	\lsgend
	
	\part $\frac{7}{3}x=14$
	\lsg[1]
\begin{align*}
\frac{7}{3}x&=14&&|: \frac{7}{3}\\
x&=14\cdot \frac{3}{7}\\
x&=6&&\\
\mathbb{L}&=\{6\}&&
\end{align*}
	\lsgend
	
		\part $\frac{4}{x}=\frac{5}{9}$
	\lsg[1]
\begin{align*}
\frac{4}{x}&=\frac{5}{9}&& \\
x&=\frac{4\cdot 9}{5}&& \\
x&=\frac{36}{5}=7,2&& \\
\mathbb{L}&=\{7,2\}
\end{align*}
	\lsgend
\end{parts}
\seitenwechsel
\question[8] Löse die folgenden Gleichungen durch Äquivalenzumformungen. Gib jeweils die Lösungsmenge an.\\ Mache \textbf{nur für b} die Probe.
\begin{parts}
	\part $2x-7+4x=3(7+2x)-5$
	\lsg[1]
	$\mathbb{L}=\{ \}$
	\lsgend
	
	\part $19+25x-15=19x+22-12x$
	\lsg[1]
	$x=1$
	\lsgend
	
	\part $2\cdot 6x+8+2x\cdot 3 = 23x-5x+8$
	\lsg[1]
	$\mathbb{L}=\mathbb{Q}$
	\lsgend
	
	\part $(18x-12):(-3)+3=13x-3(3x+5)$
	\lsg[1]
\begin{align*}
%(18x-12):(-3)+3&=13x-3(3x+5) && \\
-6x+4+3&=13x-9x-15 && \\
-6x+7&=4x-15 && |+6x+15\\
22&=10x && |:10 \Longrightarrow 
%x&=11/ 5=2,2 && \\
\mathbb{L}=\{ 2,2\} &&
\end{align*}
	\lsgend
\end{parts}
\seitenwechsel
\question[5] Stelle für das folgende Altersrätsel eine Gleichung auf und löse das Rätsel:\\
Alex ist heute doppelt so alt wie seine Schwester Britta. In 11 Jahren werden sie zusammen 46 Jahre alt sein.\\
Wie alt sind die beiden heute?
	\lsg[1]
Britta ist heute 8, Alex 16 Jahre alt.
\lsgend

\question[5] Stelle für das folgende Zahlenrätsel eine Gleichung auf und löse das Rätsel:\\
Die Summe dreier aufeinanderfolgender Zahlen ist 186. Wie heißen die drei Zahlen?
\lsg[1]
Die Zahlen heißen 61, 62 und 63.
\lsgend
\seitenwechsel
\question[5] Stelle für das folgende Geometrieaufgabe eine Gleichung auf und löse sie:\\
Ein Rechteck ist halb so breit wie lang. Der Umfang beträgt 48 cm. Wie lang sind die Seiten?
\lsg[1]
Die Rechteckseiten sind 8 cm und 16 cm lang.
\lsgend
		
\end{questions}
\label{lastpage}
\end{document}
